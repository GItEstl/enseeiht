\documentclass[12pt]{article}

\usepackage[top=3.5cm, bottom=3cm, left=2.5cm , right=2.5cm]{geometry}

\usepackage[utf8]{inputenc}
\usepackage[T1]{fontenc}
\usepackage[francais]{babel}
\usepackage{hyperref}
\usepackage{amsfonts}
\usepackage{amsmath}
\usepackage{amssymb}
\usepackage{setspace}
\usepackage{fancyhdr}
\usepackage{lipsum}
\usepackage{graphicx}

\newcommand{\Matiere}{Génie des Logiciels et des Systèmes}
\newcommand{\titre}{Modélisation, Vérification et Génération de Jeux}

\title{\Matiere:\\ \titre}
\author{Kévin CARENOU \and Thibault MEUNIER \and Matthieu PERRIER \and Sacha VANLEENE}
\date{15 décembre 2016}

\pagestyle{fancy}
\fancyhead[L]{\titre}
\fancyhead[R]{\nouppercase{\leftmark}}
\fancyfoot[L]{\Matiere}
\fancyfoot[C]{Page \thepage}
\fancyfoot[R]{ENSEEIHT - IMA 2A}

\begin{document}
\maketitle

\setcounter{page}{0}
\thispagestyle{empty} % enlever numerotation de la page de garde

\newpage

\section*{Introdution}
\lipsum[1]\lipsum[2]
\newpage

\renewcommand{\contentsname}{Sommaire}
\tableofcontents
\newpage

\section{Metamodèle}
\lipsum

\section{Editeur Graphique}
\lipsum

\section{Transformation en PetriNet}
\lipsum

\section{Implémentatiuon en Java}
Nous avons décider d'implanter ce jeu en u

\end{document}